\section{Running System-VMs} \zlabel{sec:running-system-vms}
Introduce the mechanisms that let the CloudStack management service interact with the System VMs and show how to troubleshoot problems.

\subsection{Network Configuration of System-VMs} \zlabel{subsec:network-configuration-of-system-vms}

\subsubsection{Available Networks}

\paragraph {Public Networks}
80.241.63.0/24
additional ranges may be added. Is it possible to an private range as public?? And use it dedicated for virtual router public ips.

\paragraph {Guest Networks}
Multiple like configured

\paragraph {Management Networks}
10.97.12.0/24

\paragraph {Storage Networks}
10.97.16.0/24


\paragraph {Link-local Interfaces}
Link-local connections from virtualization hosts.

\subsubsection{Network interfaces for system VMs}

For secondary storage VM:

\begin{itemize}
  \item eth0 - 169.254.0.0/16
  \item eth1 - 10.97.15.0/22
  \item eth2 - 80.241.63.0/24
  \item eth3 - 10.97.16.0/22
\end{itemize}

For console proxy VM:

\begin{itemize}
  \item eth0 - 169.254.0.0/16
  \item eth1 - 10.97.15.0/22
  \item eth2 - 80.241.63.0/24
\end{itemize}

For virtual router on isolated network:

\begin{itemize}
  \item eth0 - internal for isolated net
  \item eth1 - 169.254.0.0/16
  \item eth2 - 80.241.63.0/24 - public Network for sNAT service
\end{itemize}

For virtual router on shared network:

\begin{itemize}
  \item eth0 - IP in the shared network
  \item eth1 - 169.254.0.0/16
\end{itemize}


\subsection{root Passwords on System-VMs}

Default root password can be used to access system-VMs through virtual console.

\subsection{Communication of Management Service with System-VMs}

\begin{itemize}
  \item checkrouter.sh 
  \item ...
\end{itemize}

